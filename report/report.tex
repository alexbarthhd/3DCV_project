\documentclass[conference]{IEEEtran}
\IEEEoverridecommandlockouts
% The preceding line is only needed to identify funding in the first footnote. If that is unneeded, please comment it out.
\usepackage{cite}
\usepackage{amsmath,amssymb,amsfonts}
\usepackage{algorithmic}
\usepackage{graphicx}
\usepackage{textcomp}
\usepackage{xcolor}
\def\BibTeX{{\rm B\kern-.05em{\sc i\kern-.025em b}\kern-.08em
    T\kern-.1667em\lower.7ex\hbox{E}\kern-.125emX}}
\begin{document}

\title{Self driving toy car\\
{\footnotesize Project for 3D Computer Vision lecture, summer term 2020}
}

\author{\IEEEauthorblockN{Alexander Barth}
\IEEEauthorblockA{\textit{M. Sc. student computer engineering} \\
\textit{Heidelberg University}\\
dl248@stud.uni-heidelberg.de}
\and
\IEEEauthorblockN{Grewan Hassan}
\IEEEauthorblockA{\textit{M. Sc. student physics} \\
\textit{Heidelberg University}\\
g.hassan@stud.uni-heidelberg.de}
\and
\IEEEauthorblockN{Denis Münch}
\IEEEauthorblockA{\textit{M. Sc. student} \\
\textit{Heidelberg University}\\
denis.muench@stud.uni-heidelberg.de}
\and
\IEEEauthorblockN{Royden Wagner}
\IEEEauthorblockA{\textit{M. Sc. student computer engineering} \\
\textit{Heidelberg University}\\
royden-wagner@outlook.com}
}

\maketitle

\begin{abstract}
This report describes the self driving car toy project done in the 3D Computer Vision lecture at Heidelberg University.
The goal is to train neural networks so that the given car can drive autonomously on a track.
\end{abstract}

\begin{IEEEkeywords}
computer vision, autonomous driving, neural networks
\end{IEEEkeywords}

\section{Getting started}
For getting started an operating system needs to be flashed onto the Raspberry Pi 3 B+ which is mounted into the car.
Through the Raspberry Pi Imager the Pi OS 32-bit in release 2020-05-27 was flashed onto the SD card.
The OS is a port of Debian with the Raspberry Pi Desktop and comes with an integrated configurator to enable SSH, VNC, the camera, SPI and I2C.
To start we follow two different approaches.
The first approach uses OpenCV for classical image processing.
The second one is using machine learning algorithms.

\section{OpenCV}

OpenCV is an open computer vision library.
Controller Arduino PCA9685


\subsection{Evaluation}

\section{Conclusion}

\section*{References}

\begin{thebibliography}{00}

	\iffalse	
	Ich (Alex) hab schon mit der Art von Quellenangaben gearbeitet, finde ich persönlich relativ praktisch. Kann mich gerne ums ordnen kümmern.
	Angabe: 
	
		\bibitem{AuthorYear}
		
	Ordnen muss man manuell, das kann ich machen. 
	Am besten im gleichen Stil angeben wie das Beispiel, zur Not korrigiere ich das dann einfach, damit es einheitlich ist.
	\fi
	
\bibitem{Abadi2016} M. Abadi et al. ``Tensorflow: Large-scale machine learning on heterogeneous distributed systems,'' arXiv preprint, 2016.
\end{thebibliography}

\end{document}
